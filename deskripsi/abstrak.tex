% Mengubah keterangan `Abstract` ke bahasa indonesia.
% Hapus bagian ini untuk mengembalikan ke format awal.
\renewcommand\abstractname{Abstrak}

\begin{abstract}

  % Ubah paragraf berikut sesuai dengan abstrak dari penelitian.
  Integrasi data merupakan elemen krusial dalam menggabungkan informasi dari berbagai sumber yang berbeda ke dalam satu kesatuan terintegrasi. Terutama dalam konteks kesehatan, dashboard kinerja dan alat \emph{Business Intelligence} (BI) memainkan peran kunci dalam pengambilan keputusan yang lebih unggul dibandingkan praktik berbasis pengalaman. Namun, tantangan muncul dari sistem terpisah yang dioperasikan di setiap cabang rumah sakit, menghambat integrasi data yang diperlukan untuk dashboard kinerja efektif. Beberapa penelitian telah mengeksplorasi solusi, seperti penggunaan ETL (Extract, Transform, Load) untuk mengolah data dari berbagai sumber atau pembuatan API untuk mengakses basis data eksternal, tetapi masih ada kelemahan seperti ketergantungan pada basis data tertentu dan keterbatasan akses pada data yang tersedia di \emph{cloud}. Tugas akhir ini akan membahas pembuatan perangkat lunak berbasis desktop untuk mengintegrasikan data dari setiap cabang rumah sakit menjadi satu kesatuan dengan memanfaatkan metode ETL. Perangkat lunak harus mempertimbangkan kompleksitas data dalam sistem kesehatan, fleksibel dalam menerima data dari berbagai sumber, serta meminimalkan risiko keamanan data selama proses perpindahan data. Penggunaan metode ETL dipilih karena kemampuannya untuk menangani perbedaan format, struktur, dan jenis data dari berbagai sumber.

\end{abstract}

% Mengubah keterangan `Index terms` ke bahasa indonesia.
% Hapus bagian ini untuk mengembalikan ke format awal.
\renewcommand\IEEEkeywordsname{Kata kunci}

\begin{IEEEkeywords}

  % Ubah kata-kata berikut sesuai dengan kata kunci dari penelitian.
  Dashboard kinerja, ETL, Integrasi Data, Sistem kesehatan.

\end{IEEEkeywords}
