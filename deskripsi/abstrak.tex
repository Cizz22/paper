% Mengubah keterangan `Abstract` ke bahasa indonesia.
% Hapus bagian ini untuk mengembalikan ke format awal.
\renewcommand\abstractname{Abstrak}

\begin{abstract}

  % Ubah paragraf berikut sesuai dengan abstrak dari penelitian.
  Integrasi data adalah elemen vital dalam menyatukan informasi dari berbagai sumber, terutama dalam sistem informasi kesehatan yang mengelola data dari berbagai cabang rumah sakit. Dashboard kinerja dan alat Business Intelligence (BI) memainkan peran penting dalam pengambilan keputusan yang lebih baik dibandingkan dengan praktik berbasis pengalaman. Namun, tantangan muncul dari sistem terpisah di setiap cabang rumah sakit yang menghambat integrasi data yang efektif. Penelitian ini berfokus pada perancangan dan pembangunan perangkat lunak integrasi basis data menggunakan metode ETL. Tujuannya adalah mengembangkan perangkat lunak yang efektif dalam mengintegrasikan data dari berbagai sumber, menjaga keamanan data, dan fleksibilitas sistem. Sistem ini dirancang untuk berjalan di berbagai sistem operasi dan menangani data dari berbagai jenis basis data seperti MySQL dan PostgreSQL, sambil mempertahankan integritas referensial antar tabel yang kompleks. Perangkat lunak ini mencakup modul koneksi, modul pekerjaan sinkronisasi, dan modul eksekusi sinkronisasi, serta modul tambahan untuk keamanan. Manajemen risiko diterapkan untuk menangani kemungkinan koneksi terputus dan mendeteksi duplikat data. Sistem ini juga mengimplementasikan autentikasi kuat, enkripsi AES-256, dan protokol TLS untuk menghindari risiko keamanan data. Hasil tes menunjukkan bahwa sistem berhasil menjaga integritas data tanpa kehilangan informasi selama proses migrasi. Meski demikian, ada beberapa area untuk pengembangan lebih lanjut, seperti perluasan dukungan untuk jenis basis data tambahan dan peningkatan kemampuan multiplatform. Hasil penelitian ini memberikan kontribusi signifikan dalam menghadapi tantangan integrasi data dan menetapkan dasar untuk pengembangan sistem yang lebih komprehensif di masa depan.

\end{abstract}

% Mengubah keterangan `Index terms` ke bahasa indonesia.
% Hapus bagian ini untuk mengembalikan ke format awal.
\renewcommand\IEEEkeywordsname{Kata kunci}

\begin{IEEEkeywords}

  % Ubah kata-kata berikut sesuai dengan kata kunci dari penelitian.
  Dashboard kinerja, ETL, Integrasi Data, Sistem kesehatan.

\end{IEEEkeywords}
