% Ubah judul dan label berikut sesuai dengan yang diinginkan.
\section{Penelitian Terkait}
\label{sec:penelitianterkait}

% Ubah paragraf-paragraf pada bagian ini sesuai dengan yang diinginkan.
Penelitian sebelumnya telah banyak mengeksplorasi integrasi data dalam sistem informasi kesehatan menggunakan berbagai pendekatan dan teknologi. Salah satu penelitian yang relevan adalah oleh \citet{JAYARATNE2019996}, yang mengusulkan sebuah platform untuk integrasi data dan informasi dalam layanan kesehatan. Platform ini bertujuan untuk menyampaikan perawatan yang disesuaikan dengan individu bagi pasien dengan mengatasi heterogenitas data dan integrasi data berbasis pasien. Penelitian ini menyimpulkan bahwa platform yang diusulkan dapat menjadi langkah besar menuju pencapaian perawatan berbasis pasien (PCC) di berbagai praktik medis. Kompleksitas data dalam layanan kesehatan yang diidentifikasi dalam penelitian ini sangat relevan dengan studi ini, terutama dalam hal kesamaan studi kasus yaitu integrasi data pada layanan kesehatan dan masalah heterogenitas data pasien.

Penelitian lain yang relevan adalah oleh \citet{mutawalli2021pengembangan}, yang membahas pengembangan dashboard cerdas untuk monitoring data pasien di RSUD Praya. Penelitian ini menggunakan metode ETL dalam proses integrasi data antara dua Hospital Information System (SIM RS) di rumah sakit tersebut. Dashboard yang dikembangkan membantu rumah sakit dalam menganalisis berbagai data pasien, seperti metode pembayaran, jenis penyakit, dan distribusi pasien berdasarkan gender. Pengujian sistem menunjukkan bahwa dashboard ini cukup mendukung kinerja pemangku kepentingan di rumah sakit publik regional di Praya. Meskipun proses ETL dilakukan menggunakan aplikasi pihak ketiga, metode yang dijelaskan dalam penelitian ini memberikan referensi berharga bagi studi ini dalam hal penerapan ETL untuk integrasi data rumah sakit.

Dari penelitian-penelitian tersebut, dapat disimpulkan bahwa integrasi data dalam sistem informasi kesehatan merupakan area yang kompleks namun krusial untuk meningkatkan efisiensi dan akurasi data dalam layanan kesehatan. Metode ETL dan pendekatan platform integrasi data menawarkan solusi yang efektif, meskipun tetap terdapat tantangan dalam implementasinya. Studi ini bertujuan untuk mengatasi tantangan-tantangan tersebut dengan mengembangkan aplikasi yang dapat mengintegrasikan data secara efektif dan efisien di lingkungan rumah sakit.
