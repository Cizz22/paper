% Ubah judul dan label berikut sesuai dengan yang diinginkan.
\section{Pendahuluan}
\label{sec:pendahuluan}

% Ubah paragraf-paragraf pada bagian ini sesuai dengan yang diinginkan.

Integrasi data adalah proses menggabungkan data dari sumber-sumber yang beragam menjadi satu kesatuan yang terpadu \citet{Neang2021DataIA}. Proses tersebut menjadi fondasi utama dalam pembangunan dashboard kinerja yang efektif \citet{mutawalli2021pengembangan}. Dashboard kinerja memainkan peran krusial dalam memberikan informasi vital untuk pengambilan keputusan manajerial, terutama di sektor kesehatan. Berdasarkan penelitian \citet{Basile2023}, pemanfaatan dashboard kinerja dan \emph{Business Intelligence} (BI) dalam pengambilan keputusan dapat mengungguli praktik berbasis pengalaman dalam mengelola proses di sektor kesehatan. Selain itu, laporan yang dikeluarkan oleh Capital link terkait \emph{Performance Benchmarking Toolkit for Health Centers} menjelaskan bahwa penerapan alat analis data membantu pemimpin dalam melacak kinerja secara lebih efektif dan efisien, memahami faktor utama yang mempengaruhi, serta menggabungkan pemahaman tentang operasional untuk menjadikan pusat kesehatan lebih berkelanjutan secara finansial dan mencapai kesuksesan yang berkelanjutan \citet{CapitalLink2017}. Implementasi dashboard kinerja menjadi lebih mendesak dalam berbagai situasi kesehatan yang memerlukan pemantauan \emph{real-time} dan analisis data untuk membantu manajemen fasilitas kesehatan merespons dengan lebih cepat dan tepat.

Dalam menerapkan dashboard kinerja di bidang kesehatan, penting untuk memiliki data yang tidak hanya lengkap tetapi juga memberikan pemahaman mendalam yang diperlukan bagi operasional rumah sakit \citet{Basile2023}. Data-data ini meliputi informasi dari beragam proses bisnis di lingkungan kesehatan, mulai dari pendaftaran pasien hingga data rawat inap, jadwal praktek dokter, inventaris obat-obatan, hingga aspek administratif, keuangan, dan manajemen sumber daya manusia \citet{Basile2023}. Dengan data-data ini, dashboard kinerja dapat memberikan gambaran yang komprehensif bagi pengambilan keputusan yang efektif di dalam rumah sakit.

Berdasarkan data yang dihimpun oleh Pusat Kedokteran dan Kesehatan (Pusdokkes) polri, saat ini terdapat 57 cabang rumah sakit polri yang tersebar di berbagai daerah \citet{Aziz2023OptimalisasiPD}. Setiap cabang rumah sakit melakukan proses bisnis yang sama, yaitu melayani masyarakat dalam hal kesehatan. Dengan kegiatan layanan kesehatan yang dilakukan setiap hari, tentunya jumlah data untuk setiap cabang rumah sakit akan terus bertambah. Proses integrasi data menjadi semakin penting dalam konteks ini, karena akan memungkinkan manajemen pusat rumah sakit polri untuk melakukan analisis kualitas pelayanan kesehatan baik pada setiap cabang ataupun keseluruhan cabang dengan lebih efisien.

Namun, meskipun pentingnya pengumpulan dan integrasi data ini sangat jelas, kenyataannya mengungkapkan tantangan serius. Berdasarkan wawancara dengan salah satu pengembang dari simkes Khanza, saat ini setiap cabang rumah sakit masih mengoprasikan sistem infomasi kesehatan yang terpisah-pisah dan dijalankan secara lokal. Dampak dari hal ini adalah basis data dari setiap cabang rumah sakit belum terintegrasi dengan basis data sentral. Basis data sentral tersebut diharapkan terhubung dengan dashboard kinerja guna memantau kinerja baik keseluruhan cabang ataupun satu-persatu. Ketidaktersediaan integrasi data tersebut menjadi hambatan signifikan dalam upaya membangun dan mengimplementasikan dashboard kinerja yang efektif \citet{Oliva2018}. Untuk mengatasi hambatan ini, perlu ditemukan solusi yang memungkinkan pengumpulan dan integrasi data dari berbagai cabang rumah sakit, sehingga dashboard kinerja dapat memberikan manfaat maksimal dalam mengingkatkan kualitas dan efisiensi layanan kesehatan.

Beberapa upaya mengenai integrasi data sudah pernah dilakukan oleh \citet{Firdaus2022MEMBANGUNID} berupa implementasi desain ETL (Extract-Transform-Load) yang dapat mengolah data dari berbagai sumber dengan menggunakan Microsoft \emph{SQL Server Integration Service} untuk merancang aliran data dari sumber ke basis data tujuan. Penelitian ini memberikan hasil data yang terintegasi dengan baik dan dimuat ke dalam basis data tujuan yang telah ditentukan. Namun, meskipun upaya ini memiliki manfaat signifikan dalam menyatukan data dari berbagai sumber, terdapat kelemahan yang perlu diperhatikan. Salah satu kelemahan yang terjadi pada penelitian ini adalah penggunaan perangkat lunak pihak ketiga yang membatasi pengguna harus menggunakan basis data tertentu sebagai basis data tujuan. Penggunaan alat tertentu dapat menghambat proses integrasi disebabkan tidak kompatibelnya alat tersebut dengan infrastruktur rumah sakit.

Penelitian lain terkait integrasi data telah dilakukan oleh \cite{Herfandi_Julkarnain_Hanif_2022}, yang mengimplementasikan RESTful \emph{Application Programming Interface} (API) sebagai penghubung antara dua aplikasi pencatatan data untuk mencapai integrasi. Penelitian ini memiliki pendekatan yang berbeda, di mana tidak ada perpindahan data yang terjadi, melainkan hanya penggabungan akses pada dua basis data dalam satu aplikasi. Salah satu kelemahan yang dapat diidentifikasi pada penelitian ini adalah ketergantungan pada sumber data yang sudah harus tersedia di \emph{cloud} agar dapat diakses melalui internet. Hal ini menjadi masalah karena dalam konteks sistem informasi rumah sakit, banyak rumah sakit masih menerapkan sistem secara lokal karena kekhawatiran akan aspek keamanan data. 

Upaya yang telah diuraikan berdasarkan penelitian-penelitian sebelumnya masih belum mampu untuk menjadi solusi terhadap permasalahan integritas data pada sistem informasi kesehatan. Terlebih lagi proses migrasi data menjadi lebih rumit disebabkan heterogenitas data pada sistem informasi. Sebagaimana yang telah dijelaskan oleh penelitian yang dilakukan oleh \citet{Elamparithi2015}, migrasi data dapat menjadi tugas yang memakan waktu dan sangat mahal berbanding lurus dengan kekompleksan data; oleh karena itu, organisasi perlu menyederhanakan proses migrasi dan menjadikannya semaksimal mungkin dalam hal efisiensi biaya. Salah satu metode yang dapat digunakan untuk mengatasi tantangan integrasi data tersebut, termasuk dalam proses migrasi data, adalah Metode ETL (\emph{extract, transform, load}) \citet{Peng2023}. Metode ini melibatkan serangkaian langkah, dimulai dengan mengekstrak data dari sumber yang berbeda, kemudian mentransformasikannya agar sesuai dengan format yang diperlukan atau standar yang telah ditetapkan, dan akhirnya memuatnya ke dalam sistem atau basis data yang dituju \citet{Fana2021DataWD}. Dalam konteks sistem informasi kesehatan yang memiliki heterogenitas data, ETL menjadi penting karena dapat menangani perbedaan format, struktur, atau jenis data dari berbagai sumber informasi \citet{Peng2023}. Proses transformasi dalam ETL memungkinkan data untuk disatukan, dibersihkan, dan disesuaikan sehingga dapat diintegrasikan secara efektif. Dengan menggunakan ETL, organisasi dapat mengurangi kompleksitas serta meminimalkan biaya dan waktu yang dibutuhkan dalam proses migrasi data \citet{Fana2021DataWD}.

Selain itu ancaman kebocoran data juga menjadi hal yang perlu diperhatikan selama proses migrasi data. Data dari Ponemon Institute dalam laporan \emph{Cost of a Data Breach Report 2023} mengungkapkan bahwa biaya kebocoran data di sektor kesehatan dapat mencapai rata-rata \$11 juta \citet{Ponemon2023}. Hal ini menggarisbawahi pentingnya menjaga keamanan data selama proses migrasi data, yang dapat menjadi sangat kompleks dan rentan terhadap serangan siber.

Dari permasalahan yang telah diuraikan di atas, diperlukan suatu perangkat lunak yang dapat mengintegrasikan basis data pada sistem informasi kesehatan ke basis data sentral yang terhubung dengan dashboard kinerja dengan mempertimbangkan kekompleksan data pada sistem informasi kesehatan. Perangkat lunak tersebut fokus pada perpindahan data pada basis data lokal sistem informasi rumah sakit dengan menggunaakan metode ETL dan juga meminimalkan risiko keamanan data yang dapat terjadi ketika proses perpindahan data. Selain itu perangkat lunak juga mempunyai fleksibilitas dalam hal menerima data dari berbagai sumber basis data dan memindahkan data ke basis data sentral yang telah ditentukan.

Pembahasan pada paper ini dimulai dengan presentasi mengenai penelitian lain (Bagian \ref{sec:penelitianterkait}).
Kemudian dilanjutkan dengan penjelasan mengenai arsitektur dari sistem yang dibuat (Bagian \ref{sec:arsitektur}).
Berdasarkan hal tersebut, kami menunjukkan lorem ipsum (Bagian \ref{sec:loremipsum}).
Terakhir, didapatkan kesimpulan dari penelitian yang telah dilakukan (Bagian \ref{sec:kesimpulan}).
