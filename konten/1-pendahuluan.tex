% Ubah judul dan label berikut sesuai dengan yang diinginkan.
\section{Pendahuluan}
\label{sec:pendahuluan}

% Ubah paragraf-paragraf pada bagian ini sesuai dengan yang diinginkan.

Integrasi data merupakan proses menggabungkan data dari berbagai sumber menjadi satu kesatuan yang terpadu, menjadi fondasi utama dalam pembangunan dashboard kinerja yang efektif \cite{Neang2021DataIA, mutawalli2021pengembangan}. Dashboard kinerja memainkan peran krusial dalam memberikan informasi vital untuk pengambilan keputusan manajerial, terutama di sektor kesehatan \cite{Basile2023}. Penelitian menunjukkan bahwa pemanfaatan dashboard kinerja dan Business Intelligence (BI) dapat mengungguli praktik berbasis pengalaman dalam mengelola proses di sektor kesehatan \cite{CapitalLink2017}.

Penerapan dashboard kinerja menjadi lebih mendesak untuk pemantauan real-time dan analisis data guna membantu manajemen fasilitas kesehatan merespons dengan lebih cepat dan tepat \cite{Basile2023}. Data yang digunakan meliputi berbagai proses bisnis di lingkungan kesehatan, seperti pendaftaran pasien, rawat inap, jadwal praktek dokter, inventaris obat, hingga aspek administratif dan manajemen sumber daya manusia \cite{Basile2023}.

Pusat Kedokteran dan Kesehatan (Pusdokkes) Polri mengelola 57 cabang rumah sakit yang melakukan proses bisnis serupa dan menghasilkan data yang terus bertambah \cite{Aziz2023OptimalisasiPD}. Namun, saat ini setiap cabang masih mengoperasikan sistem informasi kesehatan yang terpisah dan dijalankan secara lokal, sehingga data belum terintegrasi dengan basis data sentral yang dibutuhkan untuk memantau kinerja \cite{Oliva2018}.

Penelitian sebelumnya mencoba mengatasi masalah ini dengan menggunakan desain ETL (Extract-Transform-Load) dan RESTful API untuk mengintegrasikan data, namun masing-masing pendekatan memiliki kelemahan terkait kompatibilitas perangkat lunak dan ketergantungan pada sumber data yang tersedia di cloud \cite{Firdaus2022MEMBANGUNID, Herfandi_Julkarnain_Hanif_2022}. Proses migrasi data yang kompleks juga menjadi tantangan, karena dapat memakan waktu dan biaya tinggi \cite{Elamparithi2015, Peng2023, Fana2021DataWD}.

Selain itu, ancaman kebocoran data perlu diperhatikan selama proses migrasi, mengingat biaya kebocoran data di sektor kesehatan yang tinggi \cite{Ponemon2023}. Diperlukan perangkat lunak yang dapat mengintegrasikan basis data lokal ke basis data sentral yang terhubung dengan dashboard kinerja, menggunakan metode ETL untuk meminimalkan risiko keamanan data dan menerima data dari berbagai sumber.


Pembahasan pada paper ini dimulai dengan presentasi mengenai penelitian lain (Bagian \ref{sec:penelitianterkait}). Kemudian dilanjutkan dengan penjelasan mengenai arsitektur dari sistem yang dibuat (Bagian \ref{sec:arsitektur}). Berdasarkan hal tersebut, kami menunjukkan lorem ipsum (Bagian \ref{sec:loremipsum}).
Terakhir, didapatkan kesimpulan dari penelitian yang telah dilakukan (Bagian \ref{sec:kesimpulan}).
